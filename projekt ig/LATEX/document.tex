\documentclass[10pt,a4paper]{article}

% ------------------------- PREAMBUŁA -------------------
\input{settings/packages}   % ścieżka względna do katalogu z ustawieniami
%\graphicspath{{images/}}    % stała ścieżka względna do katalogu z  obrazkami.



% METADATA
% DOCUMENT METADATA
\newcommand{\logoGIK}{settings/WGiK-znak.png}
\newcommand{\authorName}{Maja Jakubiec  \\ grupa 1b, Numer Indeksu: 311891}

\newcommand{\titeReport}{Projekt 1} % <<< here insert short title in the food
\newcommand{\titleLecture}{Informatyka Geodezyjna \\ sem. IV, ćwiczenia, rok akad. 2021-2022} % <<< insert title of presentation
\newcommand{\kind}{report}
%\newcommand{\mymail}{\href{mailto:name@pw.edu.pl}{name@pw.edu.pl}}
\newcommand{\supervisor}{....}
\newcommand{\gikweb}{\href{www.gik.pw.edu.pl}{www.gik.pw.edu.pl}}
\newcommand{\institut}{Zakład Geodezji Wyższej i Astronomii}
\newcommand{\faculty}{Wydział Geodezji i Kartografii}
\newcommand{\university}{Politechnika Warszawska}
\newcommand{\city}{Warszawa}
\newcommand{\thisyear}{2022}
%\date{}
% PDF METADATA
%opening
\title{GIK PW}
\author{MAJA JAKUBIEC}

\begin{document}
% ----------------------------------------------------------------
% ----------------------------  Title page
% ----------------------------------------------------------------
\begin{center}
	\rule{\textwidth}{.5pt} \\
	\vspace{1.0cm}
	\includegraphics[width=.4\paperwidth]{\logoGIK}
	\vspace{0.5cm} \\
	\Large \textsc{\titeReport}
	\vspace{0.5cm} \\  
	\large \textsc{\titleLecture}
	\vspace{0.5cm}\\
	\textsc{\authorName}  \\
	\mymail \\
	\textsc{\faculty}, \textsc{\university}  \\ 
	\city, \today
\end{center} 
\rule{\textwidth}{1.5pt}




% ---------------------------------------------------------------
% ----------------------------  Table of content
% ----------------------------------------------------------------
\tableofcontents 								% wyświetla spis treści
%\addcontentsline{to}{chapter}{Spis treści} 	% dodaje pozycję do spisu treści
% \listoffigures  								% wyświetla spis rysunków
%\addcontentsline{toc}{chapter}{Lista rysunków} % dodaje pozycję do spisu treści
% \listoftables 								% wyświetla spis rysunków
%\addcontentsline{toc}{chapter}{lista tabel}	% dodaje pozycję do spisu treści

\newpage 
\section{Rozdział 1}
W tym rozdziale też mogłoby być kilka podrozdziałów.


\subsection{Opis zadania}
Aplikacja zawierać podstawowe algorytmy:
	\item \emph Współrzedne geocentryczne (X,Y,Z) -> współrzedne geodezyjne (φ,λ,h);
	\item \emph Współrzedne geodezyjne (φ,λ,h) -> współrzedne geocentryczne (X,Y,Z);
	\item \emph Wyznaczenie współrzednych w układzie 2000wyznaczenie współrzednych w układzie 92wyznaczenie

\subsection{Podrozdział}
Opis działania programu.

Program po wgraniu do niego danych przelicza go na wymienione powyżej transformacje zapisując je w pliku txt. 

\subsection{Opis działania funkcji w programie }
\item \emph Współrzedne geocentryczne (X,Y,Z) -> współrzedne geodezyjne (φ,λ,h);
\item \emph Przelicza funkcja  xyz2plh
\item \emph Współrzedne geodezyjne (φ,λ,h) -> współrzedne geocentryczne (X,Y,Z);
\item \emph Przelicza funkcja wsp_geod2XYZ
\item \emph Wyznaczenie współrzednych w układzie 2000wyznaczenie współrzednych w układzie 92wyznaczenie
\item \emph Przelicza funkcja uklad1992 oraz u2
\newpage 
\section{Rozdział 2}
\section{Wyimportowanie programu do GitHub}


\end{document}
